\pagestyle{plain}
\begin{center}


\section*{ABSTRACT}
\addcontentsline{toc}{chapter}{Abstract} 


\end{center}

The tracking of a system's output to some goal trajectory is a common industrial problem. Repetitive tasks conducted in a controlled manufacturing environment utilize complex machinery susceptible to noise and model characteristics not captured in their design or modelling process. Iterative Learning Control (``ILC'') leverages this repetitious process in the presence of unknown, yet repeatable, disturbances to improve the output of each trial.

Constructing a controller which brings about this reduction in error is difficult to do without a complete system model. Reinforcement Learning (``RL'') provides techniques to build controllers purely from input-output data. The number of data-points needed to extract such a controller is the squared sum of the number of states and number of inputs.

When a system is translated into its ILC format, the number of effective states and inputs is scaled up by the number of steps in the manufacturing process.

Disturbances are considered to be any unknown that is constant between trials; starting position
as well as input/output noise. For linear time-invariant systems there exists a controller such that
the error between the measured output and goal output will go to zero as the number of trials
approaches infinity. A controller can be constructed from a perfect system model to satisfy state
and input penalizations. Lacking a system model makes the application of this controller much
harder as the controller must be discovered – this is where Reinforcement Learning (RL) is
introduced. As the RL problem increases in complexity with every state and output of a system in
a squared relationship, and the application of an ILC controller scales up the effective states and
inputs of a system, it is the aim of this thesis to develop methods to reduce the dimensional
representation of a system with basis functions as to facilitate the learning of an ILC Controller
with a reduced number of trials

This abstract is included in your Thesis.  It should contain: Statement of problem, procedure or method, results and conclusions.  Max 350 words.



\cleardoublepage%